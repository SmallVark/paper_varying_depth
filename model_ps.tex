\section{Modelling the Power Spectrum}
\label{sec:modelling_single_lens}
For our first analysis we will further simplify our assumptions: We imagine that all the matter between sources and observer is concentrated in a single lens plane of distance $D_{\rm d}$ from the observer. If we now distribute sources at varying distances $D_{\rm s}$, then the convergence $\kappa$ varies according to $\kappa \propto D_{\rm{ds}}/D_{\rm s}$. 
% Following \citet{1993ApJ...404..441K}, we can simulate the shear $\gamma$ from the convergence by convolution with the kernel \[
%\mathcal{D}(\bm{\theta}) = \frac{\theta_2^2-\theta_1^2-2i\theta_1\theta_2}{\pi |\bm{\theta}|^4}\, .
%\]

%Similarly, an observer that measures the shear $\gamma$ can reconstruct the convergence $\kappa$ by convolution with the kernel $\mathcal{D}^*$, where ${}^*$ denotes the complex conjugate. Assuming that the optical depth, and thus the source redshift populations, vary between pointing, an observer will measure a shear-signal that is modified by a depth-function $W(\bm{\theta}) = 1-w(\bm{\theta})$, where $\langle w(\bm{\theta})\rangle=0$ holds. The reconstruction of the observed convergence $\kappa^{\rm obs}$ will thus not yield the original convergence. In particular there is no reason why the convolution with a complex kernel should have a nonzero imaginary part. This allows us to separate our received signal into E- and B-modes following \[
%	\kappa = \kappa^E+i\kappa^B\, ,
%\]
%where $B$-Modes can not be generated by a lensing signal and are thus a tracer of remaining systematics.
%2002A&A...389..729S SRC

\subsection{Effects on the Power Spectrum}
Assuming that the depth, and thus the source redshift populations, vary between pointings, an observer will measure a shear-signal that is modified by a step-like depth-function $\gammao(\b\theta)=W(\b \theta)\gamma(\b\theta)$ with $W(\bm{\theta}) = 1+w(\bm{\theta})$, where $\langle w(\bm{\theta})\rangle=0$ holds. 
In accordance to the definition of the shear power spectrum \begin{equation}
(2\pi)^2\delta(\b\ell-\b\ell')P(|\b\ell|) = \la \hat{\gamma}(\b\ell)\hat{\gamma}(\b\ell')\ra \, ,
\end{equation}
we define the observed power spectrum via \begin{equation}
P^{\text{obs}}(\b\ell) \equiv \frac{1}{(2\pi)^2}\int \d^2 \ell'\,  \la \gammaoh(\b \ell) \gammaoh {}^*(\b \ell')\ra \, .
\end{equation}
Note that due to the depth-function both the assumptions of homogeneity and isotropy break down, which means that we can neither assume isotropy in the power spectrum, nor can we assume that $\la \gammaoh(\b \ell) \gammaoh {}^*(\b \ell')\ra$ vanishes for $\b\ell\neq\b\ell'$.
To model a constant depth on each individual pointing $\b \alpha$, we can choose random variables $w_{\b \alpha}$, that only need to satisfy $\langle w_{\b \alpha}\rangle=0$, and parametrize $w(\b\theta)$ as 
\begin{equation}
w(\b \theta) = \sum_{\b \alpha \in \mathbb{Z}^2} w_{\b \alpha} \Xi(\b \theta-L\b \alpha)\text{ , with the Box-Function } \Xi(\b \theta) = \begin{cases}
1 \qquad \b \theta\in \left[-\frac{L}{2},\frac{L}{2}\right]^2 \\
0 \qquad \text{else}
\end{cases},
\label{eq:defweightf}
\end{equation}
where $L$ is the length of one pointing.
Following the calculations in Appendix \ref{sec:calc of PS}, we derive 
\begin{equation}
P^{\text{obs}}(\b \ell)  =  P(\b \ell) + \la w^2\ra \int\frac{\text{d}^2\b k}{(2\pi)^2}\,\hat{\Xi}(\b \ell-\b k)P(\b k)\, ,
\end{equation}
where $\la w^2\ra \equiv \la w_{\b \alpha}^2\ra$ is the dispersion of the depth-function.
The observed power spectrum $P^{\text{obs}}$ is thus composed of the original power spectrum $P$, plus a convolution of the power spectrum with the fourier transfrom of a box-function, scaling with the variance of the geometric lensing efficiency $\la \frac{D_{\rm ds}}{D_{s}}\ra$. In particular, the power spectrum is not isotropic anymore. Following \citet{2002A&A...389..729S}, it would be interesting to extract E- and B-mode information out of this power spectrum, however \citet{2010A&A...520A.116S} present a more sophisticated decomposition of E- and B-modes, that can also be applied to real data using the shear correlation function, so instead we want to focus our efforts on these parts.
