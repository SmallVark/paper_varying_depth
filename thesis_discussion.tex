\chapter{Discussion and conclusion}
\label{sec:discussion}
In this thesis, we have analysed the effects of varying depth on different shear statistics. We started with the simplest case of Gaussian random fields being modified by a weight function with weights following a normal distribution, and then gradually moved on to more realistic cases. There are, however, a few simplifications that we could not get rid of. In this chapter, we will discuss the remaining assumptions and their impact on the results.
\begin{enumerate}
\item In the most general terms, we are analysing the effects of a position-based selection function on cosmic shear surveys. In our analysis, this selection function was governed by the $r$-band depth of a pointing. This neglects a number of other effects: The depth in different bands and the seeing of a pointing will also modify the number densities and redshift distributions on the scale of a pointing, whereas dithering strategies as well as imperfections in the telescope and CCD cause modifications on sub-pointing scales. However, we believe that these effects are subdominant compared to the variations caused by the $r$-band depth.

\item We have assumed an infinitely large survey area with an uncorrelated distribution of the depth-function. While the boundary effects arising from a finite survey footprint would have a small impact on the shape of the function $E(\theta)$\footnote{This would be due to the fact that a pointing next to a boundary has less neighbours, therefore making it more likely that a galaxy pair is in the same pointing.}, the governing factor is the shot noise of the depth-distribution. We have assumed that the probability that a neighbouring pointing is of percentile $n$ is exactly the expectation value, namely $1/10$. While this is true for an infinitely large survey with an uncorrelated distribution of the depth-function, Figure \ref{fig:all_xis} clearly shows that it is not valid for a $100\,\rm{deg}^2$ field. Whether our assumption holds true for the $450\,\rm{deg}^2$ footprint of the KV450 survey, or even the $1350\,\rm{deg}^2$ footprint of the final survey is an important question. Also, we have assumed an uncorrelated distribution of the depth-function, both in the simulations as well as in the models. While this is likely true as a rough approximation (very few pointings of the survey were taken on the same night and thus under the same weather conditions), effects like airmass, lunar phase, Galactic extinction and seasonal weather are likely to influence the depth on scales larger than one pointing.

These effects, however, can be mitigated: When investigating Equations \eqref{eq:pmnij_corr1} and \eqref{eq:pmnij_corr2}, we see that $P(m|n,\theta)$ denotes the probability that when one galaxy of a pair is in percentile $n$, then the second galaxy, in a different pointing, is of percentile $m$. We set this to $1/10$, but a more sophisticated approach is possible here: One can construct functions, similar to $E(\theta)$, representing the probability that the galaxy is in a neighbouring pointing in a certain direction. Then one can construct the probability distribution that this exact neighbouring pointing is of percentile $m$, given the actual survey footprint. This mitigates boundary effects, finite field effects and a correlated distribution of depth-functions and should thus be done in future analysis. An outlook to this and a few examples are given in Appendix \ref{sec:expand_eoftheta}. As a preliminary result we find that finite field effects are not significant for a $450\,\rm{deg}^2$ or $1000\,\rm{deg}^2$-field, if the distribution of depth is uncorrelated.

\item In our MCMC simulations we did not account for degeneracies with other cosmological parameters
% (apart from $\Omega_{\rm m}, \sigma_8, w_0$ and $w_a$)
 or observational effects. Especially intrinsic alignments and baryonic feedback also modify the correlation functions especially on small scales, so they are probably degenerate with the effect of varying depth. In an actual MCMC simulation that accounts for these effects, we suspect that the parameters for intrinsic alignments and baryonic feedback change to mitigate this effect, and the impact on cosmological parameters is actually smaller than in our results. Also, possible degeneracies between $S_8$ and other cosmological parameters might bias the resulting values. %Furthermore, when including the effects of dark energy we notice that our observation of $S_8$ being constant is no longer valid: For a `Euclid-like' survey we notice a bias of $\sim 0.8\sigma$. This gives rise to the assumption that the results of an MCMC with the full parameter space, including observational effects, will differ from our results. 
\end{enumerate}

Despite these repercussions, we are confident to say that the effects of varying depth are not significant for the KV450 survey. The cosmological parameters did not change significantly and the main parameter, $S_8$, is especially robust against this effect. In particular this means that a varying depth can not explain the discrepancy between observations of the local Universe and results from analysis of the CMB.

We have shown that this effect can create B-modes. However, \citet{2018arXiv181002353A} measured the B-modes of the KV450 survey in the same $\theta$-range. Those B-modes are at least one order of magnitude larger and still consistent with zero, so it is safe to say that the modes created by varying depth are negligible. An interesting observation is that the change in E-modes is as big as the created B-modes (compare Figure \ref{fig:cosebis_eandb}). This means that as soon as this effect causes significant biases in the cosmological parameters, it will also create significant B-modes\footnote{While this is no big surprise, it is not trivial. It could be possible that a systematic effect only creates E-modes and no B-modes, which would be extremely unfortunate as it could bias cosmological parameters without ever being detected by a B-mode analysis.}. Additionally, the created pattern is very characteristic, which makes it easy to recognize in a B-mode analysis of an actual survey.

For next-generation surveys like Euclid, this effect will be significant. Although Euclid is a space-based telescope, the photometric redshift determination will still be done by ground-based telescopes and therefore suffer from the same effects. While this thesis in no way serves as a quantitative study of this effect for Euclid, it does indicate that a such a study should be conducted.

As an outlook it would be interesting to perform a full MCMC simulation, including both cosmological and nuisance parameters. One might also check the most popular extensions to the standard model of cosmology, and see whether varying depth could cause a bias towards one of these models. Furthermore, the method developed in Appendix \ref{sec:expand_eoftheta} could be applied to the actual footprint of the KV450 survey to see whether finite field effects or a correlated distribution of $r$-band depth are significant.
Additionally is interesting to note that $E(\theta)$ is the azimuthal average of the function $E(\b\theta)$, which is not isotropic. Therefore, it would be possible to observe a direction-dependent correlation function $\xi_\pm^{ij,\rm{obs}}(\b\theta)$ in future surveys. An anisotropy in the observed correlation function could be a sign for the influence of varying depth.


%%% Local Variables: 
%%% mode: latex
%%% TeX-master: "../mythesis"
%%% End: 