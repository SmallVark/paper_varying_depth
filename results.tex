\section{Results}
\label{sec:results}
We applied both our methods to data from the KiDSV-450 survey and computed the ratio of observed and modeled correlation functions. Furthermore, we have conducted a simulation using ray-tracing through the millenium simulations (?)\todo{I should ask catherine about what exactly she did for the simulations or maybe even ask her to write this part of the paper?}. As in \citet{2018arXiv181206076H}, we have separated the data in 5 tomographic redshift bins and performed our analysis for a cross-correlation of all bins. The result can be seen in Figure \ref{fig:all_xis}.
	\begin{figure}
	\centering
	\includegraphics[width=1\textwidth]{images/xis_111.png}
	\caption{The ratio of observed to modeled correlation functions for the analytic method (green), the semi-analytic method (blue) and the numerical simulations (red) for a cross-correlation of all redshift bins.}
	\label{fig:all_xis}
	\end{figure}
We can see that for high redshift bins, the analytic and the semi-analytic methods are consistent, whereas for low redshift bins they significantly diverge. We explain this due to the facts that the analyic method uses simplifications that are redshift-dependent and only hold for small variations in redshift, which is not fulfilled in the low redshift bins.\todo{Should I include the plot number density vs average redshift from the talk? Maybe in the appendix to avoid too many Figures?} The simulations seem to be in rough agreement with the models, but there are some features that can not be explained. Especially we observe that the ratios of the correlation functions for large values of $\theta$ seem to consistently surpass unity, which can not be explained by our models.

As the next step we computed a reference correlation function given a fiducial cosmology for each combination of redshift bins, and modified said correlation function according to our semi-analytic model. Then we ran a Markov-Chain Monte Carlo simulation\footnote{The code for this was developed by Jan-Luca van den Busch and used in a modified version.} to check for a potential bias in the cosmological parameters, using the covariance-matrix computed in \citet{2017MNRAS.465.1454H}. As our main interest lies in the $\Omega_{\rm m}$ - $\sigma_8$ combination, we restricted our analysis to those two parameters.\todo{Maybe expand this to more parameters?}  As can be seen in Figure \ref{fig:mcmc_kids}, the impact of varying depth is noticeable, but insignificant compared to the uncertainties. However, to get a rough estimate for the impact on a Euclid-like survey, we divided the used covariance-matrix by 30, to account for the increased survey area. As can be seen in Figure \ref{fig:mcmc_euclid}, here the impact on both $\Omega_{\rm m}$ and $\sigma_8$ is significant, however it seems that the parameter $S_8$ is extremely robust against this effect.

To estimate the B-Modes created by this effect, we have extracted the \emph{Complete Orthogonal Sets of E- and B-Mode Integrals} (COSEBIS, compare \citet{2010A&A...520A.116S}), once of a reference set of correlation functions, to estimate numerical inaccuracies,\footnote{For a reference correlation function the B-Modes should be consistent with zero.} and then for the correlation functions that have been modified to account for a varying depth. We report a consistent B-Mode pattern across all redshift-bins, which can be seen in Figure \ref{fig:bmodes_cosebi}. Although we did not determine the error bars, that a KiDS-like survey would imply on those functions, \citet{2018arXiv181110596A} calculated the COSEBIs of the KiDSV-450 survey for the same range in $\theta$. The real B-Modes were about one order of magnitude higher than the ones created by the varying depth, and still found to be not significant. Given these facts, we conclude that the creation of B-Modes due to varying depth in the KiDS survey is not significant. However, the pattern seems to be very characteristic, so when one encounters B-Modes in next generation surveys, that show a similar pattern, this would suggest that they are created by a similar effect (although we can not exclude other effects that just create the same B-Mode pattern).

\begin{figure}
\centering
\includegraphics[width = \textwidth, trim = {0 1.5cm 2.5cm 0}, clip]{images/bmodes.pdf}
\caption{The $B$-modes created by a variation in survey-depth. We used the logarithmic COSEBIs described in \citet{2010A&A...520A.116S} with a range from $\theta_{\rm min} = 0.5\arcmin$ to $\theta_{\rm max} = 100\arcmin$.}
\label{fig:bmodes_cosebi}
\end{figure}
