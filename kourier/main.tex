\documentclass[a4paper]{scrartcl}
\usepackage{graphicx}

\begin{document}
We investigate the effects of a pointing-based varying depth, caused i.e. by changes in weather between observing nights, in cosmic shear surveys. We separate KV450 data into 10 percentiles, ordered by r-band depth, and extract the galaxy number density and redshift distribution for each percentile. We develop two models to estimate the impact on the shear correlation functions: A purely analytic one that solely relies on the correlations between number density and average redshift between different pointings, and a semi-analytic model that uses the exact redshift distributions of sources to calculate the change of the correlation functions. We compare our models with simulations and find that this effect causes a bias in the shear correlation functions of the order of a few percent on small scales, but the bias on the cosmological parameters is not significant. We also extract COSEBIs of the correlation functions and find that the effect can create B-modes, but not to a degree that is significant for KiDS. We find that the varying depth affects E- and B-modes to the same degree. We further discuss how to implement a correlated distribution of depth between separate pointings as well as the shape of a footprint.

We hope to register the draft with the collaboration by the end of June. The submission is planned at the beginning of August.

Attached figure: The ratio of observed to modelled correlation functions for the analytic methods (green), the semi-analytic method (blue) and the numerical simulations (red) for a cross-correlation of all redshift bins. The numbers in the upper left corners correspond to the respective redshift bins, the upper left triangle depicts the ratios of xi_+, whereas the lower right triangle depicts the ratios of xi_-.
\end{document}