\section{Introduction}
The discovery of cosmic shear has provided us with a new and powerful cosmological tool to investigate the Standard Model of Cosmology and to determine its parameters. Contrary to the analysis of the Cosmic Microwave Background (CMB) by \citet{2018arXiv180706209P}, cosmic shear is more sensitive to the properties of the local Universe and thus provides an excellent consistency check for the standard model of cosmology. Current cosmic shear surveys are especially sensitive to the parameter $S_8=\sigma_8 \sqrt{\Omega_{\rm m}/0.3}$, where $\sigma_8$ denotes the normalisation of the matter power spectrum and $\Omega_{\rm m}$ is the matter density.
It is interesting to note that all three current major cosmic shear results report a lower $S_8$ than inferred from the CMB analysis: While \citet{2018arXiv180706209P} determined a value of $S_8 = 0.830 \pm 0.013$, \citet{2018arXiv180909148H} report $S_8 = 0.800^{+0.029}_{-0.028}$ from analysis of the Subaru Hyper Suprime-Cam survey, \citet[][hereafter H18]{2018arXiv181206076H} obtained $0.737_{-0.036}^{+0.040}$ from KiDS+VIKING data and the Dark Energy Survey \citep{2018PhRvD..98d3528T} arrived at $S_8=0.782\pm 0.027$. Also, \citet{2013MNRAS.432.2433H} report $S_8 = 0.759 \pm 0.020$ from their analysis of CFHTLens data. This discrepancy has received a lot of attention \citep{2013PDU.....2..166V}. It could be interpreted as a statistical coincidence, a sign of new physics like massive neutrinos \citep{2014PhRvL.112e1303B}, time-varying dark energy or modified gravity \citep{2016A&A...594A..14P}; or as the manifestation of a systematic effect, either in the cosmic shear surveys or in the Planck mission \citep{2016ApJ...818..132A}, that is not yet accounted for. 

As weak gravitational lensing measures a tiny signal over a large sample, it is extremely sensitive to anything that systematically biases the measurements, such that the error bars in current surveys arise to equal parts from statistical and systematic uncertainties \citep[compare][]{2017MNRAS.465.1454H}. With next-generation surveys like the Large Synoptic Survey Telescope and Euclid right at the doorstep, systematic effects in gravitational lensing have received an unprecedented amount of attention \citep{2018arXiv181002353A,2019arXiv190207439B,2019arXiv190109488S}. %\todo{Citations.. \citet{2017MNRAS.464.1676A} has a few...}

To check for remaining systematics, a weak lensing signal can be divided into two components, the so-called E- and B-modes \citep{2002ApJ...568...20C,2002A&A...389..729S}. To leading order, B-modes can not be created by astrophysical phenomena and are thus an excellent test for remaining systematics. Note that the non-existence of B-modes does not necessarily imply that the sample is free of remaining systematics.

One systematic effect is the variation of depth in a survey. While effects like Galactic extinction or dithering strategies do play a role in every survey, this work focuses on the effects caused by varying atmospheric conditions, that are found in ground-based surveys. To first order, this variation can be modelled by a step-like depth function which varies from pointing to pointing. In this work we assume the specifications of the Kilo-Degree Survey, namely a collection of $1\,\text{deg}^2$ square fields. 

In Section \ref{sec:modelling_single_lens} we will introduce a simple toy model to understand this effect and analyze the impact on the power spectrum. In Section \ref{sec:xipm} we will estimate the effect on the shear correlation functions $\xi_\pm$ using two different models. We will present our results in Section \ref{sec:results}. In Section \ref{sec:discussion} we will discuss our results and comment on the impacts of our used simplifications. We will assume the standard weak gravitational lensing formalism, a summary of which can be found in \citet{2001PhR...340..291B}.
