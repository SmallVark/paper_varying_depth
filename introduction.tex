
\section{Introduction}
The discovery of cosmic shear has provided us with a new and powerful cosmological tool to investigate the $\Lambda$CDM Model and determine its parameters. Contrary to the analysis of the Cosmic Microwave Background \citep{2018arXiv180706209P}, cosmic shear is more sensitive to the properties of the local Universe and thus provides an excellent consistency check for the standard model of cosmology. Current Cosmic Shear surveys are especially sensitive to the paramter $S_8=\sigma_8 \sqrt{\Omega_{\rm m}/0.3}$, where $\sigma_8$ denotes the normalisation of the matter power spectrum and $\Omega_{\rm m}$ is the matter density.
It is interesting to note that all three current major cosmic shear results report a lower $S_8$ than inferred from CMB analysis: While \citet{2018arXiv180706209P} determined a value of $S_8 = 0.830 \pm 0.013$, 
%\citet{2013MNRAS.432.2433H} report $S_8 = 0.759 \pm 0.020$ from analysis of CFHTLens data,
\citet{2018arXiv180909148H} report $S_8 = 0.800^{+0.029}_{-0.028}$ from analysis of the Subaru Hyper Suprime-Cam survey, \citet{2017MNRAS.465.1454H} report $S_8 = 0.745\pm 0.038$ from KiDS data and the Dark Energy Survey \citep{2018PhRvD..98d3528T} reports $S_8=0.782\pm 0.027$. Also, \citet{2013MNRAS.432.2433H} report $S_8 = 0.759 \pm 0.020$ from analysis of CFHTLens data. This discrepancy could be a statistical coincidence, a sign of new physics or the manifestation of an unknown systematic effect. \todo{Citations.. \citet{2017MNRAS.464.1676A} has a few...}

To check for remaining systematics, a weak lensing signal can be divided into two components, the so-called E- and B-modes \citet{2002ApJ...568...20C,2002A&A...389..729S}. To leading order, B-modes can not be created by astrophysical phenomena and are thus an excellent test for remaining systematics. As \citet{2017MNRAS.465.1454H} report the significant detection of B-modes, it is well motivated to check for possible systematics that are yet unaccounted for. Note that the non-existence of B-modes does not imply that the sample is free of remaining systematics.

One systematic effect is the variation of depth in a survey. While effects like Galactic extinction or dithering strategies do play a role in every survey, this work focuses on the effects caused by varying atmospheric conditions, that are found in ground-based surveys. To first order, this variation can be modelled by a step-like depth function that varies from pointing to pointing. In this work we assume the specifications of the Kilo-Degree Survey, namely a collection of $1\,\text{deg}^2$ square fields. Furthermore we neglect boundary effects. 

In Section \ref{sec:modelling_single_lens} we will introduce a simple toy model to understand this effect and analyze the impact on the power spectrum. In Section \ref{sec:xipm} we will estimate the effect on the shear correlation functions $\xi_\pm$ using two different models. We will present our results in Section \ref{sec:results}. In Section \ref{sec:discussion} we will discuss our results and comment on the impacts of our used simplifications. We will assume the standard weak gravitational lensing formalism, a summary of which can be found in \citet{2001PhR...340..291B}.
