\section{Discussion}
\label{sec:discussion}
With our semi-analytic model we try to describe the impacts of varying depth in ground-based surveys. During our analysis we have assumed a few simplifications, which we will discuss.
\begin{enumerate}
\item We implemented a variation in depth that only depends on the pointing. However, in reality galactic extinction, CCD imperfections, dithering strategies and other factors also influence the depth on different scales. Although the variation between pointings is the dominating factor, a complete analysis implementing those factors could yield a slightly different result.
\item In our main analysis we assumed an infinite number of fields. This ignores boundary effects that would arise in the vincinity of the edge of a footprint (there finding a galaxy outside of the same pointing would be less likely). Also, another effect is ignored: If one has a number of $N$ fields in each percentile, and one galaxy is in a pointing of a certain percentile, then the probability that a different pointing is of the \emph{same} percentile is reduced by the factor $(N-1)/N$. However, we modeled the second effect, and already at 100 pointings it barely makes any difference. We did not model the boundary effects, but we assume that they are likewise neglible. \todo{This needs to be rewritten to account for the new insights.}
\item We have assumed that the depth of neighbouring pointings is uncorrelated. While there is no reason not to assume that, we have not yet verified that this is indeed the case.
\item In our Monte Carlo simulations we have only had $\Omega_{\rm m}$ and $\sigma_8$ as free parameters. While we do not expect most other parameters to make a difference, baryonic feedback and/or intrinsic alignments also change the correlation functions on small scales, so it is definitely possible that in a Monte Carlo chain including free parameters for these effects, those would change to mitigate the varying depth, and the impact on the actual cosmological parameters would be much smaller.
\end{enumerate}
We have determined that the effects of varying depth are not significant for the KiDSV-450 survey. For a Euclid-like survey they will play an important role, but before a model is implemented in such a precise survey, one should revise the used simplifications, especially the first one. We suspect that it should be possible to account for different aspects of varying optical depth by just modifying the function $E(\theta)$.

However, for the parameter $S_8$ this effect is insignificant, which in particular means that a variation in depth can not explain the discrepancies between the analyses of the CMB and those of the local universe. It can also be responsible for the occurence of $B$-Modes, although only to an insignificant amount.
